\documentclass[
        12pt,
        openany, %openright,
        oneside, %twoside,			%% twoside: para frente e verso ao imprimir
        a4paper,
        english,
%	french,				%% Idioma adicional
%	spanish,			%% Idioma adicional
        brazil			        %% Idioma principal
        ]{abntbibufjf}

\usepackage{lmodern}
\usepackage[T1]{fontenc}
\usepackage[utf8]{inputenc}		%% Para converter automaticamente acentos como digitados. Mude utf8 para latin1 se precisar.
                                        %% Permite digitar os acentos no teclado normalmente, sem comandos (\'e \`a , etc.).
\usepackage{lastpage}
\usepackage{indentfirst}
\usepackage{color}
\usepackage{graphicx}
\usepackage{microtype}
%% -----------------------------------------------------------------------------

%% Obs.: Alguns acentos foram omitidos.

\titulo{Uma ferramenta para recomendação de revisores de código para apoiar a colaboração em Desenvolvimento Distribuído de Software} %%Por exemplo, Titulo da tese
% \subtitulo{: subt\'itulo do trabalho}  %% Retirar o primeiro ``%'' desta linha se for utilizar subtitulo. Deixar os dois pontos antes, em ``: subt\'itulo'' .
\autor{Vinicius Junqueira Schettino}
\autorR{Junqueira Schettino, Vinicius} %%Colocar o sobrenome do autor antes do primeiro nome do autor, separados por ,
\local{Juiz de Fora}
\data{2018} %%Alterar o ano se precisar
\orientador[Orientador:]{Marco Antônio Pereira Araújo} %%Se precisar, troque [Orientador:] por [Orientadora:]
% \coorientador[Coorientador:]{Nome do coorientador } %% Retirar o primeiro ``%'' desta linha se tiver coorientador. Se precisar, troque por [Cooorientadora:].
\instituicao{Universidade Federal de Juiz de Fora}
\faculdade{Instituto de Ciências Exatas} %%Alterar, dentro de chaves {}, se precisar.
\programa{Programa de Pós-graduação em Ciência da Computação} %%Alterar, dentro de chaves {}, se precisar.
\objeto{Dissertação (Mestrado)}  %%Tese (Doutorado)
\natureza{Dissertação apresentada ao \insereprograma ~da Universidade Federal de Juiz de Fora, como requisito parcial para obtenção do título de Mestre em Ciência da Computação.}


%% Abaixo, prencher com os dados da parte final da ficha catalografica

\finalcatalog{1. Palavra-chave. 2. Palavra-chave. 3. Palavra-chave. I. Pereira Araújo, Marco Antônio, orient. II. T\'itulo.} %% Aqui fica
% escrito a palavra ``T\'itulo'' mesmo, nao o do trabalho. Se tiver coorientador, os dados ficam depois dos dados
%% do orientador (II. Sobrenome, Nome do coorientador, coorient.) e antes de ``II. T\'itulo'', o qual passa a ``III. T\'itulo''.

%% ---

\setlength{\parindent}{1.3cm}

\setlength{\parskip}{0.2cm}

\setlength\afterchapskip{12pt}


%% Iniciar o documento
\begin{document}

%% ELEMENTOS PRE-TEXTUAIS

%% Capa
\inserecapa

%% Folha de rosto
\inserefolhaderosto


%% Ficha catalografica. AO IMPRIMIR, DEIXAR NO VERSO DA FOLHA DE ROSTO.
\inserecatalog


%% Folha de aprovacao
\begin{folhadeaprovacao}

  \begin{center}
    {\chapterfont \bfseries \insereautor}

    \vfill
    \begin{center}
      {\chapterfont\bfseries\inseretitulo \inseresubtitulo}
    \end{center}
    \vfill

    \hspace{.45\textwidth}
    \begin{minipage}{.5\textwidth}
        \inserenatureza
    \end{minipage}%
    \vfill
   \end{center}

   Aprovada em: %%COLOCAR A DATA

   \begin{center} BANCA EXAMINADORA \end{center}
   \assinatura{Prof. Dr. \insereorientador \ - Orientador \\ Universidade Federal de Juiz de Fora}
%  \assinatura{Professor Dr. \inserecoorientador \ - Coorientador \\ Universidade Federal de Juiz de Fora}
   \assinatura{Professor Dr. ?? \\ Universidade ???}
   \assinatura{Professor Dr. ?? \\ Universidade ??}
%  \assinatura{...} %%RETIRE O % E PREENCHA SE PRECISAR
%  \assinatura{...}
%  \assinatura{...}
\end{folhadeaprovacao}


%% Dedicatoria. OPCIONAL. Retirar o ``%'' de cada das 4 linhas abaixo, caso queira.
% \begin{dedicatoria} \vspace*{\fill} \centering \noindent
%   \textit{ Dedico este trabalho ... (opcional)}
%   \vspace*{\fill}
% \end{dedicatoria}


%% Agradecimentos. OPCIONAL. CASO SEJA BOLSISTA, INSERIR OS DEVIDOS AGRADECIMENTOS.
\begin{agradecimentos}

De acordo com a Associa\c{c}\~ao Brasileira de Normas T\'ecnicas - 14724 (2011, p. 1) Agradecimentos
\'e o ``texto em que o autor faz agradecimentos dirigidos \`aqueles que contribu\'iram de maneira relevante \`a elabora\c{c}\~ao do trabalho.''

\end{agradecimentos}

%% Epigrafe. OPCIONAL
\begin{epigrafe}
    \vspace*{\fill}
	\begin{flushright}
		``Texto em que o autor apresenta uma cita\c{c}\~ao, seguida de autoria, relacionada com a
  mat\'eria tratada no corpo do trabalho'' \\
(ASSOCIA\c{C}\~AO BRASILEIRA DE NORMAS T\'ECNICAS, 2011, p. 2) \\
  A ep\'igrafe elaborada conforme NBR 10520 (Ep\'igrafe - Opcional)
	\end{flushright}
\end{epigrafe}


%% RESUMOS

%% Resumo em Portugu^es. OBRIGATORIO.
\setlength{\absparsep}{18pt}
\begin{resumo}
De acordo com a Associa\c{c}\~ao Brasileira de Normas T\'ecnicas - 6028 (2003, p. 2) ``o resumo deve ressaltar
o objetivo, m\'etodo e as conclus\~oes do documento (...) Deve ser composto de uma sequ\^encia de frases
concisas, afirmativas e n\~ao de enumera\c{c}\~ao de t\'opicos. Recomenda-se o uso de par\'agrafo \'unico.''
O resumo deve ter de 150 a 500 palavras.

Palavras-chave: Palavra-chave. Palavra-chave. Palavra-chave. %finalizadas por ponto e inicializadas por letra maiuscula.

\end{resumo}


%% Resumo em Ingle^s
\begin{resumo}[ABSTRACT]
 \begin{otherlanguage*}{english}
   ...

Key-words: ...
 \end{otherlanguage*}
\end{resumo}

%% Seguindo o mesmo modelo acima, pode-se inserir resumos em outras linguas.

%% Lista de ilustracoes. OPCIONAL.
\pdfbookmark[0]{\listfigurename}{lof}
\listoffigures*
\cleardoublepage


%% Lista de tabelas. OPCIONAL. Retire o ``%'' de cada das 3 linhas seguintes, caso queira.
% \pdfbookmark[0]{\listtablename}{lot}
% \listoftables*
% \cleardoublepage

%% Lista de abreviaturas e siglas. OPCIONAL
\begin{siglas} %%ALTERAR OS EXEMPLOS ABAIXO, CONFORME A NECESSIDADE
  \item[ABNT] Associa\c{c}\~ao Brasileira de Normas T\'ecnicas
  \item[UFJF] Universidade Federal de Juiz de Fora
  \item[IBGE] Instituto Brasileiro de Geografia e Estat\'istica
\end{siglas}

%% Lista de simbolos. OPCIONAL
\begin{simbolos} %%ALTERAR OS EXEMPLOS ABAIXO, CONFORME A NECESSIDADE
  \item[$ \forall $] Para todo
  \item[$ \in $] Pertence
 \end{simbolos}


%% Sumario
\pdfbookmark[0]{\contentsname}{toc}
\tableofcontents*
\cleardoublepage

%% ----------------------------------------------------------

%% ELEMENTOS TEXTUAIS

\textual
\pagestyle{simple}


\chapter{INTRODUÇÃO}  %%Nesta linha, dentro de { }, digita-se em CAIXA ALTA, como apresentado aqui

O \textit{code review} é considerado como uma das principais técnicas de diminuição de defeitos de software \cite{Boehm2001}. Nela, o autor de uma alteração na base de código de um projeto submete tal conteúdo ao crivo de um conjunto de pares técnicos, que irão revisar sua estrutura com base em um lista de regras e convenções previamente definida. Diferentes aspectos relacionados ao autor, ao revisor e ao processo de revisão em si estão diretamente relacionados à eficiência da prática. Autores relatam relação da diminuição da incidência de \textit{anti-patterns} \cite{Kemerer2009} de acordo com o nível de participação dos envolvidos e cobertura do código revisado \cite{Meneely201437, Morales2015171, Bavota201581}. Reputação \cite{Baysal2013122} e experiência \cite{Kononenko2015111} do revisor também parecem impactar nos efeitos do \textit{code review}

Intrinsecamente colaborativa, a atividade hoje é exercida com suporte de ferramentas computacionais específicas \cite{Baysal2013122}, principalmente no desenvolvimento distribuído. Dentro de workflows de trabalho descentralizados \cite{gousios2016}, a prática funciona como um \textit{gateway} de qualidade que busca garantir que apenas alterações aderentes aos padrões de qualidade do projeto serão incorporados à codebase principal. Esta etapa do desenvolvimento se torna uma oportunidade para disseminação de conhecimento, embate de ideias e discussão de melhores práticas entre profissionais de experiência e visões diferentes.

Neste contexto, porém, os os desafios à colaboração co-localizada são potencializados e as soluções tradicionais não são suficientes para fomentar esta aspecto das atividades distribuídas \cite{nicolaci2011}.
Casey \cite{casey2010} mostra que, com a distribuição geográfica dos times, diversos outros desafios, antes considerados colaterais ou resolvidos, emergem de forma a ameaçar a colaboração entre os membros da equipe: barreiras culturais, temporais e geográficas; reengenharia dos processos de desenvolvimento; resistência em compartilhar informações e conhecimento com os pares distribuídos; entre outros desafios.

Estes desafios do desenvolvimento distribuído afetam o \textit{code review} de duas formas distintas. Primeiro, o processo de revisão pode se tornar menos eficiente quando a colaboração é afetada, devido aos baixos níveis de participação e cobertura. O mesmo vale para a disseminação do conhecimento, que fica prejudicada. Outro desafio que se forma é a escolha do revisor adequado para aquele \textit{patch}. Com um vasto número de opções e pouca informação disponível sobre seus aspectos técnicos e gerenciais (e.g. tempo disponível) já que não há contato co-localizado entre eles, o a natureza distribuída deste tipo de desenvolvimento dificulta o processo de escolha do revisor, o que também pode impactar a eficiência do revisor.

Uma possível solução, visando amparar o desafio da colaboração e evitando o \textit{overhead} da escolha do revisor, seria manter grupos bem testados e experientes exercendo as atividades de revisão. Ou ainda, fixar, dentro de cada equipe de desenvolvimento,  quem são os responsáveis por revisão e pela submissão dos \textit{patches}, evitando a diversificação das relações de trabalho.

Contudo, estudos recentes demonstram que a fixação de grupos e responsabilidades pode não ser benéfica para o processo de desenvolvimento. Scott Page \cite{page2008} argumenta que a diversidade de experiências, visões e especilidades fazem com que grupos sejam mais eficientes. Já Prikladnicki et al. \cite{prikladnicki2017} apontam índicios de que a formação de grupos temporários em detrimento ou em conjunto com permanentes é um fator de eficiência em projeots de software:

\begin{description}
  ``Although old colleagues bring knowledge of the development process and prior norms from previous teams, new members bring fresh ideas that could promote project performance and creativity. Old colleagues might not do so and might not give new members a chance to implement their ideas.''
\end{description}

Expostas os desafios que o Desenvolvimento Distribuído de Software impõe sobre a escolha do revisor de código, a importância da escolha do revisor adequado do ponto de vista de colaboração e a motivação da formação de grupos heterogêneos e dinâmicos, expõe-se o objetivo deste trabalho. De acordo com a abordagem QGM (Goal/Question/Metric) proposta por Basili et al. \cite{Basili1984}: \textbf{Desenvolver} um método de recomendação de revisores \textbf{com o objetivo de} potencializar a colaboração \textbf{em relação aos aspectos} de coordenação \textbf{do ponto de vista} de revisores e autores \textbf{no contexto de} desenvolvimento distribuído de software. A principal hipótese que norteia o andamento desta proposta, e que será revisitada e discutida nos capítulos derradeiros é:

\begin{itemize}
  \item O método de recomendação apresentado pode potencializar a colaboração entre revisores e autores.
\end{itemize}

\chapter{TRABALHOS RELACIONADOS}

  \section{Revisão sistemática da literatura}

  \section{Outros trabalhos relevantes}

\chapter{MÉTODOS E FERRAMENTAS}

  \section{Ferramentas de \textit{code review}}
  (justificar o porquê da escolha do GitHub)

  \section{Métricas relevantes para recomendação do revisor}

  \section{Métricas de avaliação dos resultados}

\chapter{\textit{CODE REVIEW}}

  \section{Histórico}

  \section{Relevância}

  \section{Pull Based Method}
    (literalmente mostrar como funciona)

\chapter{SOLUÇÃO DESENVOLVIDA}
    (explicações técnicas, MER, tecnologias, etc)

\chapter{RESULTADOS}
  \section{Apresentação dos resultados}

  \section{Discussão dos resultados}

\chapter{CONCLUSÃO}

  \section{Ameaças}

  \section{Trabalhos futuros}

  \section{Considerações finais}



% a atividade de revisão remonta da décade de 80, e desde então vem evoluindo para suportar interações mais %rápidas e constantes, com uso de ferramentas computacionais e práticas ágeis. O Modern Code Review, deu lugar %à


%% ----------------------------------------------------------

%% ELEMENTOS POS-TEXTUAIS

\postextual


\bibliographystyle{abntex2-num}
\bibliography{../bibrefs/refs.bib}





%% Apendices

\begin{apendicesenv}

\chapter{Artigo Mapeamento Sistemático}

Colocar aqui o artigo.


\begin{anexosenv}


\end{anexosenv}

%%% ---
\end{document}

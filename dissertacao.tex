\documentclass[
        12pt,
        openany, %openright,
        oneside, %twoside,			%% twoside: para frente e verso ao imprimir
        a4paper,
        english,
%	french,				%% Idioma adicional
%	spanish,			%% Idioma adicional
        brazil			        %% Idioma principal
        ]{abntbibufjf}

\usepackage{lmodern}
\usepackage[T1]{fontenc}
\usepackage[utf8]{inputenc}		%% Para converter automaticamente acentos como digitados. Mude utf8 para latin1 se precisar.
                                        %% Permite digitar os acentos no teclado normalmente, sem comandos (\'e \`a , etc.).
\usepackage{lastpage}
\usepackage{indentfirst}
\usepackage{color}
\usepackage{graphicx}
\usepackage{microtype}
%% -----------------------------------------------------------------------------

%% Obs.: Alguns acentos foram omitidos.

\titulo{Uma ferramenta para recomendação de revisores de código para apoiar a colaboração em Desenvolvimento Distribuído de Software} %%Por exemplo, Titulo da tese
% \subtitulo{: subt\'itulo do trabalho}  %% Retirar o primeiro ``%'' desta linha se for utilizar subtitulo. Deixar os dois pontos antes, em ``: subt\'itulo'' .
\autor{Vinicius Junqueira Schettino}
\autorR{Junqueira Schettino, Vinicius} %%Colocar o sobrenome do autor antes do primeiro nome do autor, separados por ,
\local{Juiz de Fora}
\data{2018} %%Alterar o ano se precisar
\orientador[Orientador:]{Marco Antônio Pereira Araújo} %%Se precisar, troque [Orientador:] por [Orientadora:]
% \coorientador[Coorientador:]{Nome do coorientador } %% Retirar o primeiro ``%'' desta linha se tiver coorientador. Se precisar, troque por [Cooorientadora:].
\instituicao{Universidade Federal de Juiz de Fora}
\faculdade{Instituto de Ciências Exatas} %%Alterar, dentro de chaves {}, se precisar.
\programa{Programa de Pós-graduação em Ciência da Computação} %%Alterar, dentro de chaves {}, se precisar.
\objeto{Dissertação (Mestrado)}  %%Tese (Doutorado)
\natureza{Dissertação apresentada ao \insereprograma ~da Universidade Federal de Juiz de Fora, como requisito parcial para obtenção do título de Mestre em Ciência da Computação.}


%% Abaixo, prencher com os dados da parte final da ficha catalografica

\finalcatalog{1. Palavra-chave. 2. Palavra-chave. 3. Palavra-chave. I. Pereira Araújo, Marco Antônio, orient. II. T\'itulo.} %% Aqui fica
% escrito a palavra ``T\'itulo'' mesmo, nao o do trabalho. Se tiver coorientador, os dados ficam depois dos dados
%% do orientador (II. Sobrenome, Nome do coorientador, coorient.) e antes de ``II. T\'itulo'', o qual passa a ``III. T\'itulo''.

%% ---

\setlength{\parindent}{1.3cm}

\setlength{\parskip}{0.2cm}

\setlength\afterchapskip{12pt}


%% Iniciar o documento
\begin{document}

%% ELEMENTOS PRE-TEXTUAIS

%% Capa
\inserecapa

%% Folha de rosto
\inserefolhaderosto


%% Ficha catalografica. AO IMPRIMIR, DEIXAR NO VERSO DA FOLHA DE ROSTO.
\inserecatalog


%% Folha de aprovacao
\begin{folhadeaprovacao}

  \begin{center}
    {\chapterfont \bfseries \insereautor}

    \vfill
    \begin{center}
      {\chapterfont\bfseries\inseretitulo \inseresubtitulo}
    \end{center}
    \vfill

    \hspace{.45\textwidth}
    \begin{minipage}{.5\textwidth}
        \inserenatureza
    \end{minipage}%
    \vfill
   \end{center}

   Aprovada em: %%COLOCAR A DATA

   \begin{center} BANCA EXAMINADORA \end{center}
   \assinatura{Prof. Dr. \insereorientador \ - Orientador \\ Universidade Federal de Juiz de Fora}
%  \assinatura{Professor Dr. \inserecoorientador \ - Coorientador \\ Universidade Federal de Juiz de Fora}
   \assinatura{Professor Dr. ?? \\ Universidade ???}
   \assinatura{Professor Dr. ?? \\ Universidade ??}
%  \assinatura{...} %%RETIRE O % E PREENCHA SE PRECISAR
%  \assinatura{...}
%  \assinatura{...}
\end{folhadeaprovacao}


%% Dedicatoria. OPCIONAL. Retirar o ``%'' de cada das 4 linhas abaixo, caso queira.
% \begin{dedicatoria} \vspace*{\fill} \centering \noindent
%   \textit{ Dedico este trabalho ... (opcional)}
%   \vspace*{\fill}
% \end{dedicatoria}


%% Agradecimentos. OPCIONAL. CASO SEJA BOLSISTA, INSERIR OS DEVIDOS AGRADECIMENTOS.
\begin{agradecimentos}

De acordo com a Associa\c{c}\~ao Brasileira de Normas T\'ecnicas - 14724 (2011, p. 1) Agradecimentos
\'e o ``texto em que o autor faz agradecimentos dirigidos \`aqueles que contribu\'iram de maneira relevante \`a elabora\c{c}\~ao do trabalho.''

\end{agradecimentos}

%% Epigrafe. OPCIONAL
\begin{epigrafe}
    \vspace*{\fill}
	\begin{flushright}
		``Texto em que o autor apresenta uma cita\c{c}\~ao, seguida de autoria, relacionada com a
  mat\'eria tratada no corpo do trabalho'' \\
(ASSOCIA\c{C}\~AO BRASILEIRA DE NORMAS T\'ECNICAS, 2011, p. 2) \\
  A ep\'igrafe elaborada conforme NBR 10520 (Ep\'igrafe - Opcional)
	\end{flushright}
\end{epigrafe}


%% RESUMOS

%% Resumo em Portugu^es. OBRIGATORIO.
\setlength{\absparsep}{18pt}
\begin{resumo}
De acordo com a Associa\c{c}\~ao Brasileira de Normas T\'ecnicas - 6028 (2003, p. 2) ``o resumo deve ressaltar
o objetivo, m\'etodo e as conclus\~oes do documento (...) Deve ser composto de uma sequ\^encia de frases
concisas, afirmativas e n\~ao de enumera\c{c}\~ao de t\'opicos. Recomenda-se o uso de par\'agrafo \'unico.''
O resumo deve ter de 150 a 500 palavras.

Palavras-chave: Palavra-chave. Palavra-chave. Palavra-chave. %finalizadas por ponto e inicializadas por letra maiuscula.

\end{resumo}


%% Resumo em Ingle^s
\begin{resumo}[ABSTRACT]
 \begin{otherlanguage*}{english}
   ...

Key-words: ...
 \end{otherlanguage*}
\end{resumo}

%% Seguindo o mesmo modelo acima, pode-se inserir resumos em outras linguas.

%% Lista de ilustracoes. OPCIONAL.
\pdfbookmark[0]{\listfigurename}{lof}
\listoffigures*
\cleardoublepage


%% Lista de tabelas. OPCIONAL. Retire o ``%'' de cada das 3 linhas seguintes, caso queira.
% \pdfbookmark[0]{\listtablename}{lot}
% \listoftables*
% \cleardoublepage

%% Lista de abreviaturas e siglas. OPCIONAL
\begin{siglas} %%ALTERAR OS EXEMPLOS ABAIXO, CONFORME A NECESSIDADE
  \item[ABNT] Associa\c{c}\~ao Brasileira de Normas T\'ecnicas
  \item[UFJF] Universidade Federal de Juiz de Fora
  \item[IBGE] Instituto Brasileiro de Geografia e Estat\'istica
\end{siglas}

%% Lista de simbolos. OPCIONAL
\begin{simbolos} %%ALTERAR OS EXEMPLOS ABAIXO, CONFORME A NECESSIDADE
  \item[$ \forall $] Para todo
  \item[$ \in $] Pertence
 \end{simbolos}


%% Sumario
\pdfbookmark[0]{\contentsname}{toc}
\tableofcontents*
\cleardoublepage

%% ----------------------------------------------------------

%% ELEMENTOS TEXTUAIS

\textual
\pagestyle{simple}


\chapter{INTRODU\c{C}\~AO}  %%Nesta linha, dentro de { }, digita-se em CAIXA ALTA, como apresentado aqui

No sistema num\'erico ``a indica\c{c}\~ao da fonte \'e feita por uma numera\c{c}\~ao \'unica e consecutiva,
em algarismos ar\'abicos, remetendo \`a lista de refer\^encias ao final do trabalho,
do cap\'itulo ou da parte, na mesma ordem em que aparecem no texto.
N\~ao se inicia  a numera\c{c}\~ao das cita\c{c}\~oes a cada p\'agina.''
(ASSOCIA\c{C}\~AO BRASILEIRA DE NORMAS T\'ECNICAS, 2002, p. 4).

``O sistema num\'erico n\~ao deve ser utilizado quando h\'a notas de rodap\'e.''
(ASSOCIA\c{C}\~AO BRASILEIRA DE NORMAS T\'ECNICAS, 2002, p. 4).

``A indica\c{c}\~ao da numera\c{c}\~ao pode ser feita entre par\^enteses, alinhada ao texto, [...],
ap\'os a pontua\c{c}\~ao que fecha a cita\c{c}\~ao.''
(ASSOCIA\c{C}\~AO BRASILEIRA DE NORMAS T\'ECNICAS, 2002, p. 4).

Exemplo: Diz Rui Barbosa: ``Tudo \'e viver, previvendo.''  (15)

Um exemplo de cita\c{c}\~ao de refer\^encia \'e \cite{Ackerman1984}.
\section{SE\c{C}\~AO SECUND\'ARIA} %%Nesta linha, dentro de { }, digita-se em CAIXA ALTA

Segundo a Associa\c{c}\~ao Brasileira de Normas T\'ecnicas - 10520 (2002, p. 2)
``as cita\c{c}\~oes diretas, de at\'e tr\^e linhas, devem estar entre aspas duplas''.

A Associa\c{c}\~ao Brasileira de Normas T\'ecnicas - 10520 (2002, p. 2) determina que:
\begin{citacao}
As cita\c{c}\~oes diretas, no texto, com mais de tr\^es linhas, devem ser destacadas
com recuo de 4 cm da margem esquerda, com letra menor que a do texto utilizado
e sem aspas. No caso de documentos datilografados, deve-se observar apenas o recuo.
\end{citacao}

A UFJF disponibiliza todas as normas da Cole\c{c}\~ao ABNT atrav\'es do endere\c{c}o
\url{www.ufjf.br/biblioteca ou www.abntcolecao.com.br}

\subsection{\textbf{Se\c{c}\~ao terci\'aria}} %% O titulo da subsecao vem em negrito e caixa baixa

De acordo com a Associa\c{c}\~ao Brasileira de Normas T\'ecnicas - 14724 (2011, p. 11)
\begin{citacao}
 (...) qualquer que seja o tipo de ilustra\c{c}\~ao, sua identifica\c{c}\~ao aparece na parte superior,
 precedida da palavra designativa (desenho, esquema, fluxograma, fotografia, gr\'afico, mapa,
 quadro, retrato, figura, imagem, entre outros), seguida de seu n\'umero de ordem na ocorr\^encia
 no texto, em algarismos ar\'abicos, travess\~ao e do respectivo t\'itulo. Ap\'os a ilustra\c{c}\~ao,
 na parte inferior, indicar a fonte consultada (elemento obrigat\'orio, mesmo que seja produ\c{c}\~ao
 do pr\'oprio autor), legenda, notas e outras informa\c{c}\~oes necess\'arias \`a sua compreens\~ao (se houver).
 A ilustra\c{c}\~ao deve ser citada no texto e inserida o mais pr\'oximo poss\'ivel do trecho a que se refere.
\end{citacao}

As tabelas ``devem ser citadas no texto, inseridas o mais pr\'oximo poss\'ivel do trecho a que se
referem e padronizadas conforme o Instituto Brasileiro de geografia e Estat\'istica (IBGE).''
(ASSOCIA\c{C}\~AO BRASILEIRA DE NORMAS T\'ECNICAS, 2011, p. 11).

%%EXEMPLO PARA SE COLOCAR CORRETAMENTE A FIGURA DO ARQUIVO nomearquivo LOCALIZADA NA MESMA PASTA DESTE. %%Automaticamente,
%%envia os dados para a Lista de Ilustracoes e a ficha catalografica.

%\begin{figure}
%\begin{center}
%\caption{Informa\c{c}\~ao acima da figura}
%\includegraphics[width=7cm]{nomearquivo} \\
%\fonte{Aqui se escreve a fonte da figura}
%\end{center}
%\end{figure}

\subsubsection{\textit{Se\c{c}\~ao quatern\'aria}} %% O titulo da subsubsecao vem em italico e caixa baixa

Se for utilizado o sistema num\'erico no texto, a lista de refer\^encias deve
seguir mesma ordem num\'erica crescente. O sistema num\'erico n\~ao pode ser usado
concomitantemente para notas de refer\^encia e notas explicativas. (ASSOCIA\c{C}\~AO BRASILEIRA DE NORMAS T\'ECNICAS, 2002)

Orienta\c{c}\~oes para elabora\c{c}\~ao de refer\^encias: \\
\url{http://www.ufjf.br/biblioteca/servicos/normalizacao-2} \\
ABNT NBR 6023:1989 - \url{http://www.abntcolecao.com.br}

\subsubsubsection{Se\c{c}\~ao quin\'aria}  %% O titulo desta vem em caixa baixa

De acordo com a Associa\c{c}\~ao Brasileira de Normas T\'ecnicas - 14724 (2011) a numera\c{c}\~ao
progressiva deve conter seus t\'itulos destacados gradativamente e sua digita\c{c}\~ao deve
ser id\^entica no sum\'ario e no texto. Assim, a numera\c{c}\~ao progressiva deve ser:
\begin{citacao}
Elaborada conforme a ABNT NBR 6024. A numera\c{c}\~ao progressiva deve ser utilizada para
evidenciar a sistematiza\c{c}\~ao do conte\'udo do trabalho. Destacam-se gradativamente os
t\'itulos das se\c{c}\~oes, utilizando-se os recursos de negrito, it\'alico ou sublinhado e
outros, no sum\'ario e, de forma id\^entica, no texto. (ASSOCIA\c{C}\~AO BRASILEIRA DE NORMAS
T\'ECNICAS, 2011, p. 11).
\end{citacao}

O CDC/UFJF padronizou o destaque gradativo da numera\c{c}\~ao progressiva da seguinte forma:

$
\begin{array}{ll}
\textbf{2} & \mbox{ \textbf{SE\c{C}\~AO PRIM\'ARIA (CAIXA ALTA / COM NEGRITO)}}\\
2.1 & \mbox{ SE\c{C}\~OES SECUND\'ARIAS (CAIXA ALTA / SEM NEGRITO)}\\
2.1.1 & \mbox{ \textbf{Se\c{c}\~oes terci\'arias (Caixa baixa / com negrito)}} \\
2.1.1.1 & \mbox{ \textit{Se\c{c}\~oes quatern\'arias (Caixa baixa / com it\'alico)}} \\
2.1.1.1.1 & \mbox{ Se\c{c}\~oes quin\'arias (caixa baixa / sem negrito / sem it\'alico)}
\end{array}
$


\chapter{NOME DO CAP\'ITULO}

Texto do segundo cap\'itulo.


%% ----------------------------------------------------------

%% ELEMENTOS POS-TEXTUAIS

\postextual

%% Referencias. LISTAR EXATAMENTE AS CITADAS NO TRABALHO.
\medskip

\bibliographystyle{abntex2-num}
\bibliography{../bibrefs/refs.bib}





%% Apendices

\begin{apendicesenv}

\chapter{T\'itulo do Primeiro Ap\^endice}

``Texto ou documento elaborado pelo autor, a fim de complementar sua argumenta\c{c}\~ao,
sem preju\'izo da unidade nuclear do trabalho''
(ASSOCIA\c{C}\~AO BRASILEIRA DE NORMAS T\'ECNICAS, 2011, p. 6).

``Elemento opcional. Deve ser precedido da palavra \textbf{AP\^ENDICE}, identificado por letras mai\'usculas
consecutivas, travess\~ao e pelo respectivo t\'itulo. Utilizam-se letras mai\'usculas dobradas,
na identifica\c{c}\~ao dos ap\^endices, quando esgotadas as letras do alfabe\-to.''
(ASSOCIA\c{C}\~AO BRASILEIRA DE NORMAS T\'ECNICAS, 2011, p. 13)
\newline
EXEMPLO:
\begin{center}
\textbf{AP\^ENDICE A - Avalia\c{c}\~ao num\'erica de c\'elulas inflamat\'orias}
\end{center}


\chapter{Segundo Ap\^endice}
Texto do Segundo Ap\^endice
\end{apendicesenv}

%% Anexos

\begin{anexosenv}

\chapter{T\'itulo do Primeiro Anexo}
``Texto ou documento n\~ao elaborado pelo autor, que serve de fundamenta\c{c}\~ao,
comprova\c{c}\~ao e ilustra\c{c}\~ao''
(ASSOCIA\c{C}\~AO BRASILEIRA DE NORMAS T\'ECNICAS, 2011, p. 9).

``Elementos opcional. Deve ser precedido da palavra \textbf{ANEXO}, identificado por letras mai\'usculas
consecutivas, travess\~ao e pelo respectivo t\'itulo. Utilizam-se letras mai\'usculas dobradas,
na identifica\c{c}\~ao dos anexos, quando esgotadas as letras do alfabeto.''
(ASSOCIA\c{C}\~AO BRASILEIRA DE NORMAS T\'ECNICAS, 2011, p. 6).
\newline
EXEMPLO:
\begin{center}
\textbf{ANEXO A - Representa\c{c}\~ao gr\'afica da contagem de c\'elulas inflamat\'orias presentes
nas caudas em regenera\c{c}\~ao - Grupos de controle I (Temperatura)}
\end{center}


\chapter{T\'itulo do Segundo Anexo}
Texto do Segundo Anexo


\end{anexosenv}

%%% ---
\end{document}
